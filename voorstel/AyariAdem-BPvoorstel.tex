%==============================================================================
% Sjabloon onderzoeksvoorstel bachproef
%==============================================================================
% Gebaseerd op document class `hogent-article'
% zie <https://github.com/HoGentTIN/latex-hogent-article>

% Voor een voorstel in het Engels: voeg de documentclass-optie [english] toe.
% Let op: kan enkel na toestemming van de bachelorproefcoördinator!
\documentclass{hogent-article}

% Invoegen bibliografiebestand
\addbibresource{voorstel.bib}

% Informatie over de opleiding, het vak en soort opdracht
\studyprogramme{Professionele bachelor toegepaste informatica}
\course{Bachelorproef}
\assignmenttype{Onderzoeksvoorstel}
% Voor een voorstel in het Engels, haal de volgende 3 regels uit commentaar
% \studyprogramme{Bachelor of applied information technology}
% \course{Bachelor thesis}
% \assignmenttype{Research proposal}

\academicyear{2023-2024} % TODO: pas het academiejaar aan

% TODO: Werktitel
\title{Optimalisatie van weersvoorspellingen voor MeteoSupport: Een verkenning van traditionele numerieke modellen en integratie met Machine Learning en Deep Learning Technieken}

% TODO: Studentnaam en emailadres invullen
\author{Adem Ayari}
\email{adem.ayari@student.hogent.be}

% TODO: Medestudent
% Gaat het om een bachelorproef in samenwerking met een student in een andere
% opleiding? Geef dan de naam en emailadres hier
% \author{Yasmine Alaoui (naam opleiding)}
% \email{yasmine.alaoui@student.hogent.be}

% TODO: Geef de co-promotor op
\supervisor[Co-promotor]{N. Roose (MeteoSupport, \href{mailto:info@noodweer.be}{info@noodweer.be})}

% Binnen welke specialisatierichting uit 3TI situeert dit onderzoek zich?
% Kies uit deze lijst:
%
% - Mobile \& Enterprise development
% - AI \& Data Engineering
% - Functional \& Business Analysis
% - System \& Network Administrator
% - Mainframe Expert
% - Als het onderzoek niet past binnen een van deze domeinen specifieer je deze
%   zelf
%
\specialisation{AI \& Data Engineering}
\keywords{Meteorologie, Weersvoorspelling, Numerieke Modellen, ICON D2, GFS, HARMONIE, Machine Learning, Deep Learning, Voorspellingsalgoritmen,Tensor Calculus, Model Evaluatie, Feature Engineering, Data Preprocessing, Neurale Netwerken}

\begin{document}

\begin{abstract}
    In deze voorgestelde bachelorscriptie beoog ik de optimalisatie van weersvoorspellingen voor MeteoSupport door een grondige verkenning van traditionele numerieke weersmodellen en de potentiële integratie van geavanceerde machine learning (ML) en deep learning (DL) technieken. De centrale doelstelling is het ontwikkelen van een model dat de output van bestaande numerieke modellen, zoals ICON D2, GFS, en HARMONIE, nauwkeurig benut om verbeterde voorspellingen te genereren op zowel korte als lange termijn.
    
    \vspace{\baselineskip}
    
    De aanpak omvat een diepgaande analyse van de huidige numerieke weersmodellen, waarbij wiskundige en atmosferische wetenschappen centraal staan. Het onderzoek zal zich richten op het grondig documenteren en vergelijken van de output van deze modellen, met specifieke aandacht voor aspecten zoals atmosferische dynamica en thermodynamica.
    
    \vspace{\baselineskip}
    
    De voorgestelde methodologie omvat de ontwikkeling en implementatie van een geavanceerd ML/DL-model, waarbij technieken zoals tensor calculus, neurale netwerken en Long Short-Term Memory (LSTM) worden toegepast. Door een diepgaande evaluatie van het model en interpretatie van de resultaten, streeft deze scriptie naar een wetenschappelijk gefundeerde bijdrage aan het vakgebied van atmosferische wetenschappen en weersvoorspelling.
    
    \vspace{\baselineskip}
    
    De verwachte resultaten omvatten niet alleen een verbeterde voorspellingsnauwkeurigheid, maar ook inzichten in de interpretatie van complexe atmosferische patronen. Dit onderzoek heeft aanzienlijke meerwaarde voor MeteoSupport en andere belanghebbenden, doordat het de mogelijkheid biedt om geavanceerde modellen te implementeren die een nauwkeuriger begrip van weersverschijnselen mogelijk maken. Hierdoor kunnen cruciale beslissingen op basis van weersvoorspellingen beter worden ondersteund, met positieve gevolgen voor diverse sectoren die gevoelig zijn voor weersomstandigheden.
\end{abstract}

\tableofcontents

% De hoofdtekst van het voorstel zit in een apart bestand, zodat het makkelijk
% kan opgenomen worden in de bijlagen van de bachelorproef zelf.
\input{voorstel-inhoud}

\printbibliography[heading=bibintoc]

\end{document}